
%% bare_conf.tex
%% V1.4b
%% 2015/08/26
%% by Michael Shell
%% See:
%% http://www.michaelshell.org/
%% for current contact information.
%%
%% This is a skeleton file demonstrating the use of IEEEtran.cls
%% (requires IEEEtran.cls version 1.8b or later) with an IEEE
%% conference paper.
%%
%% Support sites:
%% http://www.michaelshell.org/tex/ieeetran/
%% http://www.ctan.org/pkg/ieeetran
%% and
%% http://www.ieee.org/

%%*************************************************************************
%% Legal Notice:
%% This code is offered as-is without any warranty either expressed or
%% implied; without even the implied warranty of MERCHANTABILITY or
%% FITNESS FOR A PARTICULAR PURPOSE! 
%% User assumes all risk.
%% In no event shall the IEEE or any contributor to this code be liable for
%% any damages or losses, including, but not limited to, incidental,
%% consequential, or any other damages, resulting from the use or misuse
%% of any information contained here.
%%
%% All comments are the opinions of their respective authors and are not
%% necessarily endorsed by the IEEE.
%%
%% This work is distributed under the LaTeX Project Public License (LPPL)
%% ( http://www.latex-project.org/ ) version 1.3, and may be freely used,
%% distributed and modified. A copy of the LPPL, version 1.3, is included
%% in the base LaTeX documentation of all distributions of LaTeX released
%% 2003/12/01 or later.
%% Retain all contribution notices and credits.
%% ** Modified files should be clearly indicated as such, including  **
%% ** renaming them and changing author support contact information. **
%%*************************************************************************


% *** Authors should verify (and, if needed, correct) their LaTeX system  ***
% *** with the testflow diagnostic prior to trusting their LaTeX platform ***
% *** with production work. The IEEE's font choices and paper sizes can   ***
% *** trigger bugs that do not appear when using other class files.       ***                          ***
% The testflow support page is at:
% http://www.michaelshell.org/tex/testflow/



\documentclass[conference]{IEEEtran}
% Some Computer Society conferences also require the compsoc mode option,
% but others use the standard conference format.
%
% If IEEEtran.cls has not been installed into the LaTeX system files,
% manually specify the path to it like:
% \documentclass[conference]{../sty/IEEEtran}





% Some very useful LaTeX packages include:
% (uncomment the ones you want to load)


% *** MISC UTILITY PACKAGES ***
%
%\usepackage{ifpdf}
% Heiko Oberdiek's ifpdf.sty is very useful if you need conditional
% compilation based on whether the output is pdf or dvi.
% usage:
% \ifpdf
%   % pdf code
% \else
%   % dvi code
% \fi
% The latest version of ifpdf.sty can be obtained from:
% http://www.ctan.org/pkg/ifpdf
% Also, note that IEEEtran.cls V1.7 and later provides a builtin
% \ifCLASSINFOpdf conditional that works the same way.
% When switching from latex to pdflatex and vice-versa, the compiler may
% have to be run twice to clear warning/error messages.






% *** CITATION PACKAGES ***
%
%\usepackage{cite}
% cite.sty was written by Donald Arseneau
% V1.6 and later of IEEEtran pre-defines the format of the cite.sty package
% \cite{} output to follow that of the IEEE. Loading the cite package will
% result in citation numbers being automatically sorted and properly
% "compressed/ranged". e.g., [1], [9], [2], [7], [5], [6] without using
% cite.sty will become [1], [2], [5]--[7], [9] using cite.sty. cite.sty's
% \cite will automatically add leading space, if needed. Use cite.sty's
% noadjust option (cite.sty V3.8 and later) if you want to turn this off
% such as if a citation ever needs to be enclosed in parenthesis.
% cite.sty is already installed on most LaTeX systems. Be sure and use
% version 5.0 (2009-03-20) and later if using hyperref.sty.
% The latest version can be obtained at:
% http://www.ctan.org/pkg/cite
% The documentation is contained in the cite.sty file itself.






% *** GRAPHICS RELATED PACKAGES ***
%
\ifCLASSINFOpdf
  % \usepackage[pdftex]{graphicx}
  % declare the path(s) where your graphic files are
  % \graphicspath{{../pdf/}{../jpeg/}}
  % and their extensions so you won't have to specify these with
  % every instance of \includegraphics
  % \DeclareGraphicsExtensions{.pdf,.jpeg,.png}
\else
  % or other class option (dvipsone, dvipdf, if not using dvips). graphicx
  % will default to the driver specified in the system graphics.cfg if no
  % driver is specified.
  % \usepackage[dvips]{graphicx}
  % declare the path(s) where your graphic files are
  % \graphicspath{{../eps/}}
  % and their extensions so you won't have to specify these with
  % every instance of \includegraphics
  % \DeclareGraphicsExtensions{.eps}
\fi
% graphicx was written by David Carlisle and Sebastian Rahtz. It is
% required if you want graphics, photos, etc. graphicx.sty is already
% installed on most LaTeX systems. The latest version and documentation
% can be obtained at: 
% http://www.ctan.org/pkg/graphicx
% Another good source of documentation is "Using Imported Graphics in
% LaTeX2e" by Keith Reckdahl which can be found at:
% http://www.ctan.org/pkg/epslatex
%
% latex, and pdflatex in dvi mode, support graphics in encapsulated
% postscript (.eps) format. pdflatex in pdf mode supports graphics
% in .pdf, .jpeg, .png and .mps (metapost) formats. Users should ensure
% that all non-photo figures use a vector format (.eps, .pdf, .mps) and
% not a bitmapped formats (.jpeg, .png). The IEEE frowns on bitmapped formats
% which can result in "jaggedy"/blurry rendering of lines and letters as
% well as large increases in file sizes.
%
% You can find documentation about the pdfTeX application at:
% http://www.tug.org/applications/pdftex





% *** MATH PACKAGES ***
%
%\usepackage{amsmath}
% A popular package from the American Mathematical Society that provides
% many useful and powerful commands for dealing with mathematics.
%
% Note that the amsmath package sets \interdisplaylinepenalty to 10000
% thus preventing page breaks from occurring within multiline equations. Use:
%\interdisplaylinepenalty=2500
% after loading amsmath to restore such page breaks as IEEEtran.cls normally
% does. amsmath.sty is already installed on most LaTeX systems. The latest
% version and documentation can be obtained at:
% http://www.ctan.org/pkg/amsmath





% *** SPECIALIZED LIST PACKAGES ***
%
%\usepackage{algorithmic}
% algorithmic.sty was written by Peter Williams and Rogerio Brito.
% This package provides an algorithmic environment fo describing algorithms.
% You can use the algorithmic environment in-text or within a figure
% environment to provide for a floating algorithm. Do NOT use the algorithm
% floating environment provided by algorithm.sty (by the same authors) or
% algorithm2e.sty (by Christophe Fiorio) as the IEEE does not use dedicated
% algorithm float types and packages that provide these will not provide
% correct IEEE style captions. The latest version and documentation of
% algorithmic.sty can be obtained at:
% http://www.ctan.org/pkg/algorithms
% Also of interest may be the (relatively newer and more customizable)
% algorithmicx.sty package by Szasz Janos:
% http://www.ctan.org/pkg/algorithmicx




% *** ALIGNMENT PACKAGES ***
%
%\usepackage{array}
% Frank Mittelbach's and David Carlisle's array.sty patches and improves
% the standard LaTeX2e array and tabular environments to provide better
% appearance and additional user controls. As the default LaTeX2e table
% generation code is lacking to the point of almost being broken with
% respect to the quality of the end results, all users are strongly
% advised to use an enhanced (at the very least that provided by array.sty)
% set of table tools. array.sty is already installed on most systems. The
% latest version and documentation can be obtained at:
% http://www.ctan.org/pkg/array


% IEEEtran contains the IEEEeqnarray family of commands that can be used to
% generate multiline equations as well as matrices, tables, etc., of high
% quality.




% *** SUBFIGURE PACKAGES ***
%\ifCLASSOPTIONcompsoc
%  \usepackage[caption=false,font=normalsize,labelfont=sf,textfont=sf]{subfig}
%\else
%  \usepackage[caption=false,font=footnotesize]{subfig}
%\fi
% subfig.sty, written by Steven Douglas Cochran, is the modern replacement
% for subfigure.sty, the latter of which is no longer maintained and is
% incompatible with some LaTeX packages including fixltx2e. However,
% subfig.sty requires and automatically loads Axel Sommerfeldt's caption.sty
% which will override IEEEtran.cls' handling of captions and this will result
% in non-IEEE style figure/table captions. To prevent this problem, be sure
% and invoke subfig.sty's "caption=false" package option (available since
% subfig.sty version 1.3, 2005/06/28) as this is will preserve IEEEtran.cls
% handling of captions.
% Note that the Computer Society format requires a larger sans serif font
% than the serif footnote size font used in traditional IEEE formatting
% and thus the need to invoke different subfig.sty package options depending
% on whether compsoc mode has been enabled.
%
% The latest version and documentation of subfig.sty can be obtained at:
% http://www.ctan.org/pkg/subfig




% *** FLOAT PACKAGES ***
%
%\usepackage{fixltx2e}
% fixltx2e, the successor to the earlier fix2col.sty, was written by
% Frank Mittelbach and David Carlisle. This package corrects a few problems
% in the LaTeX2e kernel, the most notable of which is that in current
% LaTeX2e releases, the ordering of single and double column floats is not
% guaranteed to be preserved. Thus, an unpatched LaTeX2e can allow a
% single column figure to be placed prior to an earlier double column
% figure.
% Be aware that LaTeX2e kernels dated 2015 and later have fixltx2e.sty's
% corrections already built into the system in which case a warning will
% be issued if an attempt is made to load fixltx2e.sty as it is no longer
% needed.
% The latest version and documentation can be found at:
% http://www.ctan.org/pkg/fixltx2e


%\usepackage{stfloats}
% stfloats.sty was written by Sigitas Tolusis. This package gives LaTeX2e
% the ability to do double column floats at the bottom of the page as well
% as the top. (e.g., "\begin{figure*}[!b]" is not normally possible in
% LaTeX2e). It also provides a command:
%\fnbelowfloat
% to enable the placement of footnotes below bottom floats (the standard
% LaTeX2e kernel puts them above bottom floats). This is an invasive package
% which rewrites many portions of the LaTeX2e float routines. It may not work
% with other packages that modify the LaTeX2e float routines. The latest
% version and documentation can be obtained at:
% http://www.ctan.org/pkg/stfloats
% Do not use the stfloats baselinefloat ability as the IEEE does not allow
% \baselineskip to stretch. Authors submitting work to the IEEE should note
% that the IEEE rarely uses double column equations and that authors should try
% to avoid such use. Do not be tempted to use the cuted.sty or midfloat.sty
% packages (also by Sigitas Tolusis) as the IEEE does not format its papers in
% such ways.
% Do not attempt to use stfloats with fixltx2e as they are incompatible.
% Instead, use Morten Hogholm'a dblfloatfix which combines the features
% of both fixltx2e and stfloats:
%
% \usepackage{dblfloatfix}
% The latest version can be found at:
% http://www.ctan.org/pkg/dblfloatfix




% *** PDF, URL AND HYPERLINK PACKAGES ***
%
%\usepackage{url}
% url.sty was written by Donald Arseneau. It provides better support for
% handling and breaking URLs. url.sty is already installed on most LaTeX
% systems. The latest version and documentation can be obtained at:
% http://www.ctan.org/pkg/url
% Basically, \url{my_url_here}.




% *** Do not adjust lengths that control margins, column widths, etc. ***
% *** Do not use packages that alter fonts (such as pslatex).         ***
% There should be no need to do such things with IEEEtran.cls V1.6 and later.
% (Unless specifically asked to do so by the journal or conference you plan
% to submit to, of course. )


% correct bad hyphenation here
\hyphenation{op-tical net-works semi-conduc-tor}


\begin{document}
%
% paper title
% Titles are generally capitalized except for words such as a, an, and, as,
% at, but, by, for, in, nor, of, on, or, the, to and up, which are usually
% not capitalized unless they are the first or last word of the title.
% Linebreaks \\ can be used within to get better formatting as desired.
% Do not put math or special symbols in the title.
\title{Directed Research: Semantic Visual Odometry}


\author{\IEEEauthorblockN{Sai Ramana Kiran Pinnama Raju}
	\IEEEauthorblockA{Email: spinnamaraju@wpi.edu}}



% use for special paper notices
%\IEEEspecialpapernotice{(Invited Paper)}




% make the title area
\maketitle

\section{Introduction}

Visual Odometry (VO) and Simultaneous Localization and Mapping (SLAM) are an integral component in several robotic and wearable technologies such as Unmanned Aerial Vehicles (UAV), Autonomous Ground Vehicles (AGV), Augmented (AR) and Virtual Reality (VR) headsets. With the advancements in silicon chips and faster NN computation, data driven approaches to these fields have gained more traction among research communities. Classical approaches to VO or SLAM usually involves tracking pixel intensities using optical flow or extracting low level features. Although these approaches perform decently well in several datasets, the features are short-term and often needs to be recomputed under different exposures or camera settings. These limitations can be reduced by tracking high level information instead. This write-up is a summary of related work, current problems and some potential ways to solve.

\section{Towards ``Robust'' Visual Odometry}
A visual odometry component is vital in several applications and hence it is important to clearly define what is considered as ``Robust'' Visual odometry. A philosophical non-exhaustive list towards this definition would be the following
\begin{enumerate}
    \item \label{itm:partial} Robust towards partial or blinded sensor measurements
    \item \label{itm:track_env}Tracking across different environment conditions and motion conditions (example: high speed)
    \item \label{itm:faster_exec}As close to real time as possible in execution 
    \item \label{itm:occulusions}Identifying and tracking in case of occlusions
    \item \label{itm:static_dynamic}Demarcating static and dynamic objects in the field
    \item \label{itm:no_drift}Minimal drift over a long run
    \item \label{itm:cheap_compute}Cheaper to compute on smaller robots and devices
\end{enumerate}

In this work, I am trying to address points \ref{itm:track_env},\ref{itm:faster_exec},\ref{itm:occulusions} using semantic features of the scene and deep neural networks. The following section expands on the related work that has been done so far on the points of interest.

\section{Related Work}
Most of the traditional approaches, are classified into direct and indirect methods. Direct methods use the pixel intensities and construct optical flow optimization problems. These methods are computationally expensive and hence slower since the density of data needs to be processed is high, however the odometry decisions are based on more information making it more reliable. Some of the popular works like \cite{DSO} take this method a bit further by sampling the pixels across frames and optimize the photometric error, thereby decreasing the computational requirements. 


On the other hand, indirect methods like \cite{orbstereo} primarily track corners and other geometric features for tracking. These methods are computationally inexpensive compared to direct methods but suffer from viewpoint variances. Moreover, their odometry information is based on very selective data points from the frames and might not be as reliable as direct methods output.

Irrespectively, both of these methods need hand tailored feature selection and needs different parameter or optimizer selection based on different scenarios. Moreover, these algorithms are working based off of low-level features making them unreliable for aspects mentioned in \ref{itm:track_env},\ref{itm:occulusions}. Modern techniques approach this problem by making use of data points collected over the years and generalize the solution by adding more ``intelligence'' to the system. Some of the current research use DNNs in the following ways,

\begin{enumerate}
    \item \label{itm:enhance} Enhancing low-level feature tracking with semantic information \cite{SalientDSO},\cite{DS-SLAM},\cite{VSO} 
    \item Dynamic object masks using semantic networks \cite{RGBDSlam}
    \item \label{itm:end_to_end} End-to-end pose estimation of camera \cite{semantic_autonomous_object},\cite{DeepVO},\cite{UnDeepVO}
    \item Tracking using semantic networks alone \cite{semantic only},\cite{semantic only vehicle}
    \item Embedding memory into the pipelines \cite{memory and refine},\cite{adaptive memory}
    \item Using semantic maps to construct full 3D maps \cite{SemanticSLAM},\cite{Dense3d}
\end{enumerate}

\section{Problems of Interest}
On a high level, current research in \ref{itm:enhance} strike a good balance of enhacing robustness of features by mapping them to a semantic label. This improves their view invariance and tracking. However, some of these methods do not perform well in real time and are highly dependent on semantic labelling. Methods in \ref{itm:end_to_end} are interesting note, since they perform the entire camera pose estimation using neural networks. Inspiring by this methodology, we can train different aspects like geometric features, optical flow independently and combine them together for better explainability and robustness. Works like \cite{DeepVIO} expand on this end-to-end approach by incorporating both stereo and inertial measurements in network training. Another way to expand this could be is integrating visual transformer based networks along with inertial sensors to further improve the odometry accuracy




DeepVIO
out of box, 
simulator -- imu, vision

calibrated camera -- self calibration
IMU 
domain of the network
domain invariance or adaption 
data creation or 
real time or accuracy 
embed or RTX -- novelty


% Now, even though these methods are computationally efficient and fast, they are bound with limitations like tracking loss due to high speeds, lighting changes, occlusion and active object. To account for all or some these factors, many recent works are now exploring the philosophy of adding high level semantic inferences to one of the modules in VO or SLAM. Within these some of the them like  focus on  using semantic information for tracking and robustness while some others like \cite{SemanticSLAM},\cite{Dense3d} focus on constructing 3D maps. More recent works like ,\cite{semantic_autonomous_object} employ Deep Neural Networks for extracting






% trigger a \newpage just before the given reference
% number - used to balance the columns on the last page
% adjust value as needed - may need to be readjusted if
% the document is modified later
%\IEEEtriggeratref{8}
% The "triggered" command can be changed if desired:
%\IEEEtriggercmd{\enlargethispage{-5in}}

% references section

% can use a bibliography generated by BibTeX as a .bbl file
% BibTeX documentation can be easily obtained at:
% http://mirror.ctan.org/biblio/bibtex/contrib/doc/
% The IEEEtran BibTeX style support page is at:
% http://www.michaelshell.org/tex/ieeetran/bibtex/
%\bibliographystyle{IEEEtran}
% argument is your BibTeX string definitions and bibliography database(s)
%\bibliography{IEEEabrv,../bib/paper}
%
% <OR> manually copy in the resultant .bbl file
% set second argument of \begin to the number of references
% (used to reserve space for the reference number labels box)
\begin{thebibliography}{1}


\bibitem{DSO}
Engel, J., Koltun, V. and Cremers, D., 2017. Direct sparse odometry. IEEE transactions on pattern analysis and machine intelligence, 40(3), pp.611-625.

\bibitem{orbstereo}  
R. Mur-Artal, J. D. Tardos, ``ORB-SLAM2: an Open-Source SLAM system for monocular stereo and RGB-D cameras'', 2016.


\bibitem{SalientDSO}
H. -J. Liang, N. J. Sanket, C. Fermüller and Y. Aloimonos, "SalientDSO: Bringing Attention to Direct Sparse Odometry," in IEEE Transactions on Automation Science and Engineering, vol. 16, no. 4, pp. 1619-1626, Oct. 2019, doi: 10.1109/TASE.2019.2900980.

\bibitem{DS-SLAM}
Yu, C., Liu, Z., Liu, X.J., Xie, F., Yang, Y., Wei, Q. and Fei, Q., 2018, October. DS-SLAM: A semantic visual SLAM towards dynamic environments. In 2018 IEEE/RSJ International Conference on Intelligent Robots and Systems (IROS) (pp. 1168-1174). IEEE.

\bibitem{VSO}
K.-N. Lianos, J. L. Schonberger, M. Pollefeys, and T. Sattler, “VSO: Visual
semantic odometry,” in Proc. Eur. Conf. Comput. Vis., 2018, pp. 234–250.

\bibitem{SemanticSLAM}
McCormac, J., Handa, A., Davison, A. and Leutenegger, S., 2017, May. Semanticfusion: Dense 3d semantic mapping with convolutional neural networks. In 2017 IEEE International Conference on Robotics and automation (ICRA) (pp. 4628-4635). IEEE.

\bibitem{Dense3d}
Hermans, A., Floros, G. and Leibe, B., 2014, May. Dense 3d semantic mapping of indoor scenes from rgb-d images. In 2014 IEEE International Conference on Robotics and Automation (ICRA) (pp. 2631-2638). IEEE.

\bibitem{RGBDSlam}
J. Liu, X. Li, Y. Liu and H. Chen, "RGB-D Inertial Odometry for a Resource-Restricted Robot in Dynamic Environments," in IEEE Robotics and Automation Letters, vol. 7, no. 4, pp. 9573-9580, Oct. 2022, doi: 10.1109/LRA.2022.3191193.

\bibitem{semantic_autonomous_object}
Qian, Z., Patath, K., Fu, J. and Xiao, J., 2021, May. Semantic slam with autonomous object-level data association. In 2021 IEEE International Conference on Robotics and Automation (ICRA) (pp. 11203-11209). IEEE.

\bibitem{DeepVO}
Wang, S., Clark, R., Wen, H. and Trigoni, N., 2017, May. Deepvo: Towards end-to-end visual odometry with deep recurrent convolutional neural networks. In 2017 IEEE international conference on robotics and automation (ICRA) (pp. 2043-2050). IEEE.

\bibitem{UnDeepVO}
Li, R., Wang, S., Long, Z. and Gu, D., 2018, May. Undeepvo: Monocular visual odometry through unsupervised deep learning. In 2018 IEEE international conference on robotics and automation (ICRA) (pp. 7286-7291). IEEE.

\bibitem{DeepVIO}
Han, L., Lin, Y., Du, G. and Lian, S., 2019, November. Deepvio: Self-supervised deep learning of monocular visual inertial odometry using 3d geometric constraints. In 2019 IEEE/RSJ International Conference on Intelligent Robots and Systems (IROS) (pp. 6906-6913). IEEE.

\bibitem{semantic only}
H. Mahé, D. Marraud and A. I. Comport, "Semantic-only Visual Odometry based on dense class-level segmentation," 2018 24th International Conference on Pattern Recognition (ICPR), 2018, pp. 1989-1995, doi: 10.1109/ICPR.2018.8545263.

\bibitem{semantic only vehicle}
M. Herb et al., "Semantic Image Alignment for Vehicle Localization," 2021 IEEE/RSJ International Conference on Intelligent Robots and Systems (IROS), 2021, pp. 1124-1131, doi: 10.1109/IROS51168.2021.9636517.

\bibitem{memory and refine}
Xue, F., Wang, X., Li, S., Wang, Q., Wang, J. and Zha, H., 2019. Beyond tracking: Selecting memory and refining poses for deep visual odometry. In Proceedings of the IEEE/CVF Conference on Computer Vision and Pattern Recognition (pp. 8575-8583).

\bibitem{adaptive memory}
Xue, F., Wang, X., Wang, J. and Zha, H., 2020. Deep visual odometry with adaptive memory. IEEE Transactions on Pattern Analysis and Machine Intelligence.


\end{thebibliography}




% that's all folks
\end{document}


